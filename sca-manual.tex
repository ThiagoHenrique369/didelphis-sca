%!TEX program = xelatex
\documentclass[10pt,letterpaper]{article}
\usepackage{fontspec}
\usepackage{xunicode}
% Ok apparently you should never load xunicode before fontspec
\usepackage{verbatim}
\usepackage[left=1.25in,right=1.25in,top=1.25in,bottom=1.25in]{geometry}

\newcounter{excounter}
% verbatim example
\newenvironment{vex}[1]{
	\refstepcounter{excounter}
	\noindent\emph{Ex.} (\arabic{excounter}\label{#1})
	\verbatim
}{\endverbatim}

\title{Haedus Toolbox SCA, Manual}
\author{Samantha Fiona Morrígan McCabe}
\date{\today}

\setmainfont[Ligatures=Common]{Linux Libertine O}
\setsansfont{Linux Biolinum O}
\setmonofont[Scale=0.8]{DejaVu Sans Mono}

\begin{document}
\maketitle
\tableofcontents

\section*{Introduction} 
\label{sec:introduction}
The Haedus Toolbox SCA is a powerful, script-driven sound-change applier program implemented in Java. It supports UTF-8 natively and has a rule syntax designed for clarity and similarity to sound change notation in linguistics.

The Haedus SCA supports capabilities like multiple rule conditions, regular expressions, metathesis, unrestricted variable naming, and scripting but also allows novice users to ignore advanced functionality they do not need.

This manual is divided into two main parts, plus an appendix. The first part provides a walkthrough of the SCA and it's capabilities using a completed example rules file included with the program; the second part is a more detailed reference to the rule language and its syntax; the appendix provides details of implementation which are only likely to be of interest to more advanced users.

% section: introduction (end)
% -----------------------------------------------------------------------------

\section{User Guide} 
\label{prt:user_guide}
The following sections will guide you, as a new user, through using the SCA and create a rules script, from prerequisites and executing provided example scripts, to writing basic rules, to using some of the more powerful supported features.

% part: user_guide (end)
% -----------------------------------------------------------------------------

\subsection{Setup and Execution}
\label{sec:setup_and_execution}
This section describes the steps required to set up and run the Haedus SCA, either on Windows or Unix-like systems. The SCA is provided as an executable \texttt{.jar} file, which will require that the Java Runtime Environment (JRE 1.6 or later) be installed on your system.

% section: setup_and_execution (end)
% -----------------------------------------------------------------------------

\subsubsection{Java Runtime Environment}
\label{sub:java_runtime_environment}
If you do not have a JRE installed, you will need to acquire an appropriate version from Oracle's website. If you have a JRE installed already, it must be Java 6 or later (because that version was released quite a few years ago at the time this manual is being prepared, it likely will be).

If everything is set up correctly, when you open a terminal window and type \texttt{java -version} you should see output like the following:\\
\begin{vex}{ex:javatest}
samantha@colossus: java -version
java version "1.7.0_79"
OpenJDK Runtime Environment (IcedTea 2.5.5) (7u79-2.5.5-0ubuntu0.14.04.2)
OpenJDK 64-Bit Server VM (build 24.79-b02, mixed mode)
\end{vex} 
\noindent
On a Windows or Mac system, you will see something a bit different (specifically, the JRE implementation), but the ``\texttt{java version}'' itself is what matters. If the \texttt{java} command is not recognized, your environment variables may not be set correctly, either because your \texttt{\$JAVA\_HOME} is not set correctly, or because Java is not available on your system path.

% subsection: java_runtime_environment (end)
% -----------------------------------------------------------------------------

\subsubsection{Running the Sound-Change Applier}
\label{sub:running_sca}
%In order to actually run the Haedus SCA, your best option is to navigate a terminal to the folder where you have extracted the archive.

The Haedus SCA is run from the command line, using the \texttt{toolbox.bat} or \texttt{toolbox.sh} commands (for Windows and Unix-like systems respectively). It is possible to run the \texttt{.jar} file directly, but because it include the version number, this is rather more cumbersome.

In either case, run the SCA, you only need to provide the paths of one or more rules files: \\
\begin{vex}{ex:runstandard}
toolbox.sh script1.rule script2.rule ... scriptN.rule
\end{vex}
\noindent
There are no other arguments to provide, only the script names. Rather than specifying input and output lexicons when starting the program, lexicons are read and written within the rules files themselves (section \ref{sub:scripting_capabilities}).

% subsection: running_sca (end)
% -----------------------------------------------------------------------------

\subsection{Writing Rule Scripts}
\label{sec:writing_rule_scripts}
All of the SCA's functionality is controlled using script files. In addition to rule and variable definitions, the scripts may also contain commands controlling the reading and writing of lexicons, setting normalization modes, and importing or executing commands from other script files. This section will introduce you to most of the supported features in the context of the provided example file \texttt{pie-pkk.rule}, which represents the rules for converting the provided Proto-Indo-European word list into a corresponding list for Kuma-Koban; the hope is that this will be a smooth way of introducing new users to the rule syntax. At this point and for the rest of the section, some of the examples will refer to the provided \texttt{pie-pkk.rule} example file.

The Haedus rule language is generally insensitive to whitespace, and while items in lists (sets, items in rule transforms, or variable definitions) are delimited by whitespace, it is still insensitive to quantity\footnote{Specifically, lists are stripped of padding whitespace at the beginning and end, and split using the regex \texttt{\textbackslash\hspace{0pt}s+}}; this will be apparent in the examples. 

Comments can always be created using the \texttt{\%} symbol, as shown below:

\begin{vex}{ex:comment}
% This is a comment (full-line)
ph th kh > f s x % this is a rule, followed by a comment (inline)
\end{vex}
\noindent
Anything following the \texttt{\%} will be ignored; there is not currently any way to create block comments, however. 

% section: writing_rule_scripts (end)
%------------------------------------------------------------------------------

\subsubsection{Using Normalization} 
\label{sub:using_normalization}
Though it is by no means required that you set this yourself, the SCA's support for normalization modes can be a powerful tool. Not specifying any mode will leave all inputs untouched by default. To set a mode, use the \texttt{mode} keyword, followed by one of the three supported modes, \texttt{decomposition}, \texttt{composition}, and \texttt{intelligent}.

The first two perform canonical decomposition, canonical decomposition followed by canonical composition, respectively, according to the Unicode standard.\footnote{UAX 15: Unicode Normalization Forms ``http://unicode.org/reports/tr15/''} The short explanation is that that decomposition will take composed Unicode characters with diacritics, and separate the base and diacritic characters in a consistent way, replacing a single composed character like ö with o+ ̈. Composition does this as well, but then attempts to re-assemble them, if there are any composed characters available. For a sound-change applier, decomposition may be the most reliable, as it will ensure that you can always write rules which operate on the diacritics themselves.

% ◌

The intelligent segmentation mode is a bit different from either of these; it is superficially similar to \texttt{composition} but rather than using pre-composed Unicode characters, it uses Unicode character classes to distinguish between base and diacritic characters, and assemble strings it can treat as if they were single characters. This allows the SCA to treat \emph{q̇ʷʰ}, for instance, as a single character, rather than four. In brief, when it finds a non-diacritic character followed by one or more diacritics, they will be attached to the non-diacritic.

The details of this segmentation algorithm are described in section \ref{sub:segmentation}, but if you develop a sense of how it works, it can make writing rules much simpler than they might otherwise be. One major advantage is that, using the \texttt{intelligent} mode, a rule which targets \emph{k} will not trigger on \emph{kʰ}. Of course, if you desire that behavior, then you can choose another normalization mode, or use none at all.

% subsection using_normalization (end)
%------------------------------------------------------------------------------

\subsubsection{Loading Lexicons}
\label{sub:loading_lexicons}
One of the first things you will want to do when writing rules is to load a lexicon. This is done using the \texttt{open} keyword, followed by the path to a lexicon in single or double quotes, the optional\footnote{The \texttt{as} keyword was made optional because its presence can make the commands more fluent and clear to read, but it is not syntactically important, so it may be omitted for compactness} keyword \texttt{as} and a file-handle name by which you can reference the lexicon later. In the provided example file, you are loading \texttt{pie\_lexicon.txt} and binding it to the file-handle \texttt{LEXICON}, using the command

\begin{vex}{ex:openlexicon}
open 'pie_lexicon.txt' as LEXICON
\end{vex}

You can use \texttt{close} or \texttt{write} to write the lexicon's state to disk; \texttt{close} is the same as \texttt{write}, but also removes the file handle and lexicon from memory. It may be good practice to write the output commands as soon as you open a lexicon, to ensure you don't forget later. Both these commands use a similar, if reversed, syntax to loading lexicons, with the file-handle first, then the output path. You might start writing a rules file like the following: \\
\begin{vex}{ex:closelexicon}
open 'pie_lexicon.txt' as LEXICON
% Intervening rules go here
write LEXICON as 'late-pie_lexicon.txt'
% Intervening rules go here
close LEXICON as 'pkk_lexicon.txt'
\end{vex}
\noindent
Writing intermediate lexicons can be a useful way of debugging during development, and ensuring that your outputs look correct at intermediate stages.
%\footnote{Author's note: I've created a ticket for future versions to support ``probe words'', where the user can bind an input word to a handle, and assert that it has a particular value at any number of points in the rule script. This could (optionally) halt processing entirely, and write the lexicons to disk in their current state to aid the user in identifying any errors. See [toolbox-sca-15] https://github.com/samanthamccabe/toolbox-sca/issues/51}

Finally when opening a lexicon from disk, it will be read into memory and normalized using whatever mode was last defined. If you were to change modes \emph{after} loading a lexicon, it could lead to unexpected behavior. For example, if no mode is set when a lexicon is loaded, but intelligent segmentation is enabled prior to writing any rules, a rule containing \texttt{pʰ} will not trigger on a word containing \texttt{pʰ} because it will have been loaded as two separate characters \texttt{p} and \texttt{ʰ}. 

% subsection loading_lexicons (end)
%------------------------------------------------------------------------------

\subsubsection{Reserving Characters} 
\label{sub:reserving_characters}
Though there are no uses of reserving characters in the Kuma-Koban examples, it can be useful to do, especially when you are not using intelligent segmentation (see section \ref{sub:using_normalization}), or where your orthographic conventions may conflict in some way with it (such as if your language uses pre-aspirated stop, which you represent \emph{ʰt}).

You can use the \texttt{reserve} keyword to indicate which sequences of characters which are intended to represent single sounds. Following the keyword, simply list the string you wish to be treated this way, separated by whitespace:

\begin{vex}{ex:reserve}
reserve ph th kh kw kwh
\end{vex}
\noindent

This will ensure that a rule intended to target \emph{p} will not affect \emph{ph} and a rule intended to target \emph{kw} will not affect \emph{kwh}. Be mindful when writing rules; carelessly reserving sequences could lead to unexpected behavior in your outputs.

% section: reserving_characters (end)
% -----------------------------------------------------------------------------

\subsubsection{Writing Simple Rules}
\label{sub:writing_simple_rules}
It is possible to write some rather complicated rules using the Haedus language, but it is generally not necessary. Most rules are fairly simple, and these are what will be discussed here.

This is first rule you will see in the Kuma-Koban example:\\
\begin{vex}{ex:simplerule}
y w > i u
\end{vex}
\noindent
The intention is to change every \emph{y} to an \emph{i} and every \emph{w} to a \emph{u}, regardless of context. The first set of rules in the Kuma-Koban example are like this; simple rules for making orthographic corrections in the raw data.

Though it is used less often, deletion is a useful feature of rules. The following example shows how to do this:

\begin{vex}{ex:simplerule3}
% Delete morpheme boundary marking
- > 0
\end{vex}
\noindent
It will delete all hyphens in the lexicon (\emph{i.e.} remove morpheme boundaries). Changing any segments to \texttt{0} cause them to be deleted. 

%There are two more early rules that should also be instructive, and illustrate how to work with basic conditions and word boundaries, and how to delete segments, respectively. More detailed information is discussed in section \ref{sub:rules}

%This first rule is intended to apply to universally so it does not require a condition. 

%The rule in ex. \ref(ex:simplerule2) uses, as part of its condition, the word-boundary symbol \texttt{\#} which is a rather common component of many condition. The rule itself can be understood as meaning ``\emph{n-} changes to \emph{nˌ} word-initially''.


% section: writing_simple_rules (end)
% -----------------------------------------------------------------------------

\subsubsection{Defining Variables}
\label{sub:defining_variables}
A common early task in the development process is defining variables you expect to use often. The assignment operator \texttt{=} binds a variable name on the left to a set of values on the right; values can be either terminal symbols, or other variables. The following is a representative example from the Kuma-Koban data:\\
\begin{vex}{ex:basicvars}
H = x ʔ
N = m n
L = r l
R = N L
W = y w
\end{vex}
\noindent
Variables can be defined at any point in the script, and can also be re-defined at any point. Once a variable is defined a certain way, it will have that value for every subsequent reference until and unless it is changed. You can find more detailed information in section \ref{sub:variables}.

% section defining_variables (end)
% -----------------------------------------------------------------------------

\subsubsection{writing Conditions}
\label{sub:writing_conditions}
While the rules we've discussed so far have not needed to use conditions, the  majority of rules might ever find yourself writing will. One of the simplest possible conditions is seen in the following rule:\\
\begin{vex}{ex:simplerule2}
% For correct handling of negation prefix
n- > nˌ / #_
\end{vex}
\noindent

If you are familiar with the standard formalisms for discussing sound change academically, most of this notation will be familiar. The forward-slash \texttt{/} indicates the start of the condition, and the underscore \texttt{\_} indicates where the ``source'' symbol (any of the elements on the left side of the \texttt{>} operator) occurs in relation to the rest of the condition.

%Good conditions are where an SCA's real power lies. In this rule language, there is a lot to know about writing conditions, and that is detailed in section \ref{sub:conditions}; however, we will discuss these in a more organic fashion, as they appear in the Kuma-Koban example, for the benefit of new users.

There is a very common rule in Indo-European languages where \emph{*e} is changed to \emph{*a} in the environment of \emph{*h₂} and possibly \emph{*h₄} if there ever was such a thing\footnote{where distinguishing between these is not possible, \emph{*hₐ} is used, at least in those sources which admit a \emph{*h₄}}. For convenience, these are simply converted to consonants \emph{x} and \emph{ʕ}.

Expressed verbally, we might say ``\emph{e} changes to \emph{a} before or after \emph{x} or \emph{ʕ}''. One way to write this is with four rules:\\
\begin{vex}{ex:fourlaryngealrule}
E > A / _x
E > A / _ʕ
E > A / x_
E > A / ʕ_
\end{vex}
\noindent
If this seems sub-optimal, there are two ways we can condense the number of individual rules. First, we can cut the number of rules in half by using \emph{sets}:\\
\begin{vex}{ex:laryngealrulesets}
E > A /      _{x ʕ}
E > A / {x ʕ}_
\end{vex}
\noindent
A set is just a list of elements, separated by whitespace, and contained within a pair of curly braces \texttt{\{} and \texttt{\}}\footnote{Padding is also permitted around the set elements, so writing \texttt{\{ x ʕ \}} is legal and equivalent to \texttt{\{x ʕ\}}}. Functionally, it's nearly the same as defining these in a variable beforehand, but without the need to actually use a separate command to do it.

There is another tool we can use to combine both rules into one, and that is the \texttt{or} keyword:

\begin{vex}{ex:fourrules}
% e to a, before or after x or ʕ
E > A / {x ʕ}_ or _{x ʕ}
\end{vex}
\noindent
Using \texttt{or} allows a rule to have multiple conditions, any one of which can trigger the rule. If find you have the same change being made under several separate conditions, you can write then as a single rule using \texttt{or}. But only do this judiciously, as is can substantially impair the legibility of your rules, especially if the conditions are complicated.

% section writing_conditions (end)
% -----------------------------------------------------------------------------

\subsubsection{Writing Advanced Conditions} 
\label{sec:writing_advanced_conditions}

% section writing_advanced_conditions (end)
% -----------------------------------------------------------------------------

% >>>>>

%% OLD WRITING-RULES SECTION --------------------------------------------------

%\subsection{Writing Rules}\label{sec:writingrules}

%Transformation rules are the heart of any SCA. The Haedus SCA has a lot of power, and is designed to make basic tasks quite simple. It gives you the ability to easily change multiple sequences with a single rule, and under multiple alternate conditions, to use regular expressions in conditions, and to perform metathesis using some more advanced features.

%The SCA uses the right-angle-bracket, or greater-than symbol \texttt{>} as the transformation operator separating the sounds you would like to change on the left, from the sounds to which you want those to change on he right:

%\begin{vex}{ex:rulebasic}
%a₁ a₂ a₃ a₄ > e₁ e₂ e₃ e₄
%\end{vex}
%\noindent
%Each element on the left corresponds to one on the right, so that \texttt{a₁} is changed to \texttt{e₁}, \texttt{a₂} to \texttt{e₂} and so on. In the context of the SCA language, this part of the rule is called the \emph{transform}, the left side of which is the \emph{source} and the right side the \emph{target}.

%The rule in example (\ref{ex:rulebasic}) is unconditioned and will apply any time one of the source symbols is encountered. Adding a condition is quite similar in this SCA as in others: it is set off from the transform with the forward slash \texttt{/} and uses the underscore \texttt{\_} to denote where in the pattern the source symbol should occur, as in:

%\begin{vex}{ex:rulecondition}
%d > r / _#
%\end{vex}
%\noindent
%which indicates that \emph{d} changes to \emph{r} at the end of a word, \emph{i.e.} when it occurs before a word boundary, designated by \texttt{\#}. A marginally more complex rule changes voiceless plosives to voiced plosives between vowels can be written:
%
%\begin{vex}{ex:rulevoicing}
%V = a i u
%p t k > b d g / V_V
%\end{vex}

%\subsection{Scripting}\label{sec:scripting}

% It may not be appropriate to deal with these topics here; they are likely to be adequately described in the syntax section
%\subsection{Back-Reference and Indices}\label{sec:indices}
%\subsection{Using Phonetic Features}\label{sec:usingfeatures}

% PART II =====================================================================

\section{Scripts \& Syntax}
\label{prt:scripts_and_syntax}
This section describes the commands supported by the SCA, and their syntax ans semantics. It attempts to be as detailed as possible with informative examples and, in some cases, provides notes on implementation.

Scripts and lexicons are, by default, read in as UTF-8; it is not currently possible to change this. One substantial difference between this SCA and others is that these rule files are \emph{compiled} rather than merely interpreted; as script commands are read in, they are validated and parsed to objects in memory. This has several advantages, namely that because compilation happens once, rules are not repeatedly re-interpreted for each word; it also allows that errors can be caught immediately at compile time, rather than at runtime.

In this rule language generally, the contents of lists are whitespace-separated (the space character, or tab) and quantity-insensitive (one space is treated the same as two), so you can use extra spaces or tabs to make columns align, as you will see throughout the examples. This is also true of the padding around most operator symbols (\emph{viz.} \texttt{= > /})

%While whitespace is used to separate items in lists, padding around operators and delimiters is optional. As elsewhere the quantity is not important.

Script files may contain comments, starting with \texttt{\%}, and may be placed at the start of a line, or in-line; in either case, anything to the right of the comment symbol is ignored.

The following characters have special meanings in the SCA script language and should only be used in the contexts they are expected:

\begin{vex}{ex:reserved}
% # $ * ? + ! ( ) { } [ ] 0 . \_ = / >
\end{vex}
\noindent
Most of these are sensitive to context, but \texttt{=}, \texttt{/}, and \texttt{>} are restricted and using them inappropriate ways will causethe script to  fail to compile.\footnote{This is because these symbols in particular are used to identify variable definitions and transformation rules. Specifically, a line which contains \texttt{=} assumed by the parser to be a varaible definition. Likewise, a line which contains \texttt{>} is assumed to be a rule definition.} If a section indicates that a symbol on this list is allowed, then it is allowed in that context but should still be avoided in others.

% section: scripts_and_syntax
% -----------------------------------------------------------------------------

\subsection{Scripting Capabilities}
\label{sec:scripting_capabilities}
In addition to the standard functions provided by the SCA, the script syntax also allows the user to do things like read and write lexicons, import or execute other script files, and set normalization and formatting modes within a rules file.

% subsection: scripting_capabilities
% -----------------------------------------------------------------------------

\subsubsection{Reading \& Writing Lexions}
\label{sub:reading_and_writing_lexicons}
%\small{\emph{This functionality is not available in basic mode.}}
When operating the SCA in standard mode, these commands are used to read and write lexicons from disk. Once a lexicon is in memory, any sound changes run in the script will be applied to all open lexicons. The three available commands are \texttt{open}, \texttt{write} and \texttt{close} commands. 
%
To open a lexicon, use the \texttt{open} command in the following way:

\begin{vex}{ex:open}
open "language.lex" as LANGUAGE
\end{vex}

\noindent
Lexicons are referenced by a file-handle, \texttt{LANGUAGE} in ex. \ref{ex:open}. The handle name must begin with a capital letter an must contain only capital letters, numbers, or the underscore.

The difference between \texttt{write} and \texttt{close} is that the former will write the lexicon, in it's current state, to the specified location, but the handle will still be available and future changes will be applied; \texttt{close} will also write the lexicon to disk but remove the it from memory, making the file-handle unavailable. These commands have the same syntax, simply substituting \texttt{write} for \texttt{close} in the following:

\begin{vex}{ex:close}
close LANGUAGE as "new_language.lex"
\end{vex}
\noindent
Lexicons are not automatically written closed when the script completes, so if you open lexicons and forget to close them, their changes will be lost.

% subsubsection: reading_and_writing_lexicons (end)
% -----------------------------------------------------------------------------

\subsubsection{Import \& Execute Commands}
\label{sub:import_and_execute_commands}
%\emph{This functionality is not available in basic mode.} 
It is possible to use other script files using the \texttt{import} and \texttt{execute} commands. Using \texttt{import} will read the contents of another rule file, and insert it's contents into that position in the script and compiles them. Using \texttt{execute} will compile and run all these commands in the file immediately. The syntax is simple and is as follows:

\begin{vex}{ex:commands}
execute "other1.rule"
import  "other2.rule"
\end{vex}

\noindent
The key difference is that \texttt{execute} will run the script separately (reading and writing lexicons, applying rules, calling other resources, and so on) while \texttt{import} places the script into your current script so that any lexicons or variables specified in the other file will be usable in the current script.

% section: import_and_execute_commands (end)
% -----------------------------------------------------------------------------

\subsubsection{Normalization \& Formatting}
\label{sub:normalization_and_formatting}
The Haedus SCA provides the capability of normalizing input data, in the lexicons, rules, and feature models. Four modes are provided.

The default is \texttt{none} which will not alter your inputs in any way. The modes \texttt{composition} and \texttt{decomposition} will perform canonical composition or decomposition respectively according to the Unicode standard.
% Insert reference to standard
The SCA also provides an \texttt{intelligent} mode, which applies canonical decomposition and then uses a series of rules based on Unicode character classes and character ranges to attach diacritics to the appropriate head character. Details of the implementation are described in the appendix, section \ref{sub:segmentation}.

%If running the SCA in basic mode, normalization is set using the \texttt{-m} flag, followed by one of \texttt{C D S}, for \texttt{composition}, \texttt{decomposition}, and \texttt{smart} respectively. If none is given, \texttt{none} is used by default:
%
%\begin{vex}{ex:normalizationbasic}
%toolbox -b -mS input.lex basic.rule output.lex
%\end{vex}

To set the formatter while running in standard mode, use the keyword \texttt{normalize} followed by a supported mode:

\begin{vex}{ex:normalization}
normalize intelligent
\end{vex}

Because the formatting mode has an effect on the parsing of all forms of data, it is critical that you declare the format before loading any other resources or declaring any rule or variable statements.

It is possible to chanage the formatting mode anywhere in a rule file, but it is suggested that you not do this without an extremely good understanding of how this will impact the parsing of your data. Once a resource is loaded or statement declared in a given formatting mode, it will not be affected by future mode changes.

% subsubsection: normalization_and_formatting (end)
% -----------------------------------------------------------------------------

\subsection{Variables}
\label{sec:variables}
The SCA allows for the definition of variables (and re-definition) on-the-fly, anywhere in the script. Variables definitions consist of a label, the assignment operator \texttt{=} and a space-separated list of values. Representative example is shown below.

\begin{vex}{ex:variables}
TH = pʰ tʰ kʰ
T  = p  t  k
D  = b  d  g
W  = w  y  ɰ
N  = m  n
C  = TH T D W N r s
\end{vex}

The definitions in example \ref{ex:variables} illustrate several things: using whitespace to align symbols into columns in a convenient and readable way, reasonably free variable label naming, and the use of variables in the definition of other variables.

There are no formal restrictions placed on variable lablels, beyond requiring that they not use characters reserved by the SCA. You will notice here that both \texttt{TH} and \texttt{T} are defined. This is possible becayse when SCA parses a rule or variable definition, it searches for variables by finding the \emph{longest} matching label first. If you have variables \texttt{T}, \texttt{H}, and \texttt{TH}, a rule containing the string \texttt{TH} will always be understood by the SCA to represent the variable \texttt{TH}, and not \texttt{T} followed by \texttt{H}. The best way to avoid this situation is to name variable carefully.\footnote{Though I do not see this as a problem in need of a resoltion, I will note that this conflict, should it arise at all, is most likely to do so in a rule condition. In that context, it is possible to simply use the regular expression language (see section \ref{sec:expressions}) to your advantage by wrapping one or both variables in parentheses to avoid the conflict: \texttt{(T)(H)}}

It is possible to re-assign variables at any point in the script.This includes appending values to an existing variable in the following way:

\begin{vex}{ex:variables2}
C = C h
\end{vex}

%%TODO: INCORPORATE ------------------------------------- 
%You can also append or prepend new values onto an existing variable in the following way: \\
%\begin{vex}{ex:reassignvars}
%C = t k
%C = C q % Append
%C = p C % Prepend
%\end{vex}
%\noindent
%This is especially helpful for variables like \texttt{C} used for consonants or plosives, because sound changes might add new consonants to the language's inventory.


% Using features instead; OR using variables as aliases for feature names

% subsection: variables (end)
% -----------------------------------------------------------------------------

\subsection{Rules}
\label{sec:rules}
This section describes the syntax of rule definitions, and related functionality like conditions and regular expressions. Any uncommended line containing the right angle-bracket symbol \texttt{>} will be parsed as a rule; that is, a line represents a rule definition \emph{iff} it contains the symbol \texttt{>}.

%allowing symbols within each item of a lexicon to be conditionally transformed in a uniform way. The rule format is flexible and its complexity is commensurate with its power; a rule might be as simple as

A rule must consist minimally of three parts: one symbol on the left-hand-side, the \texttt{>} operator, and one symbol on the right-hand-side, as in the following:

\begin{vex}{ex:simplerule}
u > y
\end{vex}
\noindent
This will change every instance of \emph{u} to \emph{y} without exception. This part of the rule (left- and right-hand-sides and the operator is the \emph{transform}, as distinguished from the \emph{condition}, which occurs following the \texttt{\/} symbol. Rules which do not include a condition are effectively a bare transform. It is far more common for rules to include conditions, which are discussed in detail in section \ref{sec:conditions}.

%\begin{vex}{ex:complexrule}
%GH > G / _{R W}?VV?C*GH
%\end{vex}

% subsection: rules (end)
% -----------------------------------------------------------------------------

\subsubsection{Transformation}
\label{sub:transformation}
Either the left- and right-hand-sides of the transformation must contain the same number of elements, or the right must contain exactly one; if the right hand side contains a single segment, this signals the SCA to change each sequence of the left to the one on the right. This can be a useful way of representing mergers. The following statements are equivalent:

\begin{vex}{ex:convergence}
ɑ e o > a a a
ɑ e o > a
\end{vex}

\noindent
Note that in this case \texttt{ɑ e o > a a} will produce a compilation error.

The right-hand side of the transformation is permitted to contain the literal zero \texttt{0} which represents a deleted segment. For example, the following rule will delete schwas where they occur in word-final position:

\begin{vex}{ex:deletion}
ə > 0 / _#
\end{vex}

\subsubsection{Indices \& Backreferences}
\label{sub:indices_and_backreferences}
Within the transform of a rule, it is possible to use indexing in the target to refer to symbols in the source, which can be very useful in writing commands for metathesis or total assimilation. Within the rule's target group, the dollar sign \texttt{\$} followed by a digit allows you to refer back to a variable matched within the source group. For example, the commands

\begin{vex}{ex:backreferences}
C = p t k
N = n m
CN > $2$1
\end{vex}

\noindent
allow us to easily represent metathesis, swapping \texttt{N} and \texttt{C} wherever \texttt{N} is found following \texttt{C}. 

When SCA parses a rule, it keeps track of each variable in the source part of the transform and knows in the above example, that \texttt{C}  is at index \texttt{1} and \texttt{N} is at index \texttt{2}. The target part of the transform lets us refer back to this using the \texttt{\$} symbol and the index of the variable we wish to refer to.

We can actually go slightly further and use the indices on a \emph{different} variable, provided they have the same number of elements. In a variation on the previous example, we can write

\begin{vex}{ex:indices}
C = p t k
G = b d g
N = n m
CN > $2$G1
\end{vex}

\noindent
which does the same as the above, but also replaces any element of \texttt{C} with the corresponding element of \texttt{G}. So, if a word is \emph{atna}, the rule will change it to \emph{anda}.

This can also be used for some kinds of assimilation and dissimilation, such as simplifying clusters of plosives by changing the second to be the same as the first:

\begin{vex}{ex:assimilation}
C = p t k
CC > $1$1
\end{vex}

\noindent
This will change a word like \emph{akpa} to \emph{akka}. %in this case, it is actually equivalent to write \texttt{CC > C\$1}
To assimilate in the other direction, you can simply use \texttt{\$2\$2}

% subsubsection: indices_and_backreferences (end)
% -----------------------------------------------------------------------------

\subsection{Conditions}
\label{sec:conditions}
The rule condition determines where in a word it will be possible for a rule to apply. The condition is set off from the rule transform by the forward-slash symbol \texttt{/}. Following this delimiter, the underscore character \texttt{\_} is required. It can be preceded and followed by expressions (either of which may be blank) which define where in a word the transformation may be applied.\footnote{At the implementation-level, this is done by searching a word, start to end, for a symbol from the transform. If it is found, the symbols before and after it are checked against the condition. If both match, the changes are applied.}

These can include literal characters, regular expressions, and the word-boundary character \texttt{\#}. % Include mention of feature definitions for future releases.
Many common conditions are relatively simple, representing things like ``between consonants'', ``before nasal consonants'', or ``word-initially'', as in the following example:

\begin{vex}{ex:devoicing}
b d g > p t k / #_
\end{vex}
\noindent
which represents devoicing of plosivses in word intitial position.

\subsubsection{Expressions}
\label{sub:expressions}
Within rule conditions, it is possible to use regular expressions. While the Haedus implementation is not POSIX compliant, it nevertheless permits the use of metacharacters \texttt{?}, \texttt{*}, \texttt{+}, and \texttt{.} and also allows grouping through with parentheses \texttt{()}.

These have the behaviors you should expect if you are familiar with regular expression languages: \texttt{?} indicates that a preceding expression may be matched zero or one times; \texttt{*} indicates that a preceding expression may be matched zero or more times; \texttt{+} indicates that a preceding expression may be one or more times. The dot-metacharacter \texttt{.} will match any single literal. Grouping with parentheses allow application of the quantifying metacharacters to sequences of expressions, such as: 

\begin{vex}{ex:grouping}
(ab?c)+
\end{vex}
\noindent
which will match the expression \texttt{ab?c} one or more times, and thus any of the following strings \emph{ac}, \emph{abc}, \emph{acac}, \emph{abcac}, \emph{abcacacabac} and so on.\footnote{As a product of the syntax of groups, it is possibly to nest them to any depth, so that \texttt{(a)} will match the same set of strings as \texttt{(((a)))} for example, but doing this has no benefit and unnecessary nesting merely generates more complicated machines which require more time to evaluate.}

%As in other regular expression languages, a single symbol is an expression, as is a sequence of expressions, or a group or set.

The most substantial deviation from common regular expression implementations is the use of curly braces to represent sets, though this is consistent with their use in phonology and mathematics. Expressions enclosed in curly braces are separated by spaces, so that the following expression:

\begin{vex}{ex:sets}
{ E₁ E₂ E₃ }
\end{vex}
\noindent
will match any one of the expressions \texttt{E₁}, \texttt{E₂}, or \texttt{E₃}. Sets can be used with quantifier metacharacters, just as groups can.

% subsection: conditions
% -----------------------------------------------------------------------------

\subsubsection{Condition Chaining} 
\label{sub:condition_chaining}
When a single transformation is triggered under multiple conditions, you may use chaining to represent this more concisely, without the the need to define the same rule twice.
% subsubsection condition_chaining (end)
% -----------------------------------------------------------------------------

%\subsection{Phonetic Features} 
%\label{sub:phonetic_features}
% subsection phonetic_features (end)
% -----------------------------------------------------------------------------

\section{Appendix}\label{sec:appendix}

\subsection{Intelligent Segmentation}\label{sec:segmentation}

%\subsection{Recursive Nondeterministic Finite-State Automata}
%\label{sub:rndfa}

%\subsection{Multivalue Articulatory Feature-Model}
%\label{sub:mafm}
\end{document}
